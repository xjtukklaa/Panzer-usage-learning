\subsubsection{Tutorial Step 1}
Kokkos的练习代码包括两个部分,第一部分为原始的没有改为并行的
代码,第二部分为改为并行后的修正代码,在此将原始代码保存为注释
,修改后代码为正常运行代码。
\begin{lstlisting}
#include <limits>
#include <cmath>
#include <cstdio>
#include <cstdlib>
#include <cstring>
#include <iostream>
#include <Kokkos_Core.hpp>

// 检查命令行输入的N,M,S是否符合要求,nrepeat表示其重复计算次数
void checkSizes(int &N, int &M, int &S, int &nrepeat);

int main(int argc, char *argv[])
{
    int N = -1;
    int M = -1;
    int S = -1;
    int nrepeat = 100;

    // 命令行参数读取
    for (int i = 0; i < argc; i++)
    {
    if ((strcmp(argv[i], "-N") == 0) || (strcmp(argv[i], "-Rows") == 0))
    {
        N = pow(2, atoi(argv[++i]));
        std::cout << "  User N is " << N << std::endl;
    }
    else if ((strcmp(argv[i], "-M") == 0) || (strcmp(argv[i], "-Columns") == 0))
    {
        M = pow(2, atof(argv[++i]));
        std::cout << "  User M is " << M << std::endl;
    }
    else if ((strcmp(argv[i], "-S") == 0) || (strcmp(argv[i], "-Size") == 0))
    {
        S = pow(2, atof(argv[++i]));
        std::cout << "  User S is " << S << std::endl;
    }
    else if (strcmp(argv[i], "-nrepeat") == 0)
    {
        nrepeat = atoi(argv[++i]);
    }
    else if ((strcmp(argv[i], "-h") == 0) || (strcmp(argv[i], "-help") == 0))
    {
        std::cout << "y^T*A*x Options: " << std::endl;
        std::cout << "-Rows (-N) <int>:      rows (default: 4096) " << std::endl;
        std::cout << "-Columns (-M) <int>:   columns (default: 1024) " << std::endl;
        std::cout << "-Size (-S) <int>:      matrix (default: N*M ) " << std::endl;
        std::cout << "-nrepeat <int>:        repeats (default: 100) "<< std::endl;
        std::cout << "-help (-h):            print this message " << std::endl;
        exit(1);
    }
    }

    // 检查参数是否合规
    checkSizes(N, M, S, nrepeat);

    // Kokkos环境初始化,和MPI_Initialize用法一致,所以代码都需要在
    // Initialize和Finalize之间
    Kokkos::initialize(argc, argv);
    {
    // 检测可用并行空间
    // EXERCISE give-away: Choose an Execution Space.
    // using ExecSpace = Kokkos::Serial;
    // using ExecSpace = Kokkos::Threads;
    // using ExecSpace = Kokkos::OpenMP;
    // using ExecSpace = Kokkos::Cuda;

    // EXERCISE: Choose device memory space.
    // using MemSpace = Kokkos::HostSpace;
    // using MemSpace = Kokkos::CudaSpace;
    // using MemSpace = Kokkos::CudaUVMSpace;

    #ifdef KOKKOS_ENABLE_CUDA
    #define MemSpace Kokkos::CudaSpace
    #endif

    #ifdef KOKKOS_ENABLE_HIP
    #define MemSpace Kokkos::Experimental::HIPSpace
    #endif

    #ifdef KOKKOS_ENABLE_OPENMPTARGET
    #define MemSpace Kokkos::OpenMPTargetSpace
    #endif

    #ifndef MemSpace
    #define MemSpace Kokkos::HostSpace
    #endif

    // 并行空间
    using ExecSpace = MemSpace::execution_space;
    
    // EXERCISE give-away: Use a RangePolicy.
    // using range_policy = Kokkos::RangePolicy<ExecSpace>;
    // 执行策略,在此模板只指定了并行空间
    using range_policy = Kokkos::RangePolicy<ExecSpace>;

    // EXERCISE give-away: Choose a Layout.
    // EXERCISE: When exercise is correctly implemented, then
    //           either layout will generate the correct answer.
    //           However, performance will be different!

    // using Layout = Kokkos::LayoutLeft;
    using Layout = Kokkos::LayoutRight;

    // Allocate y, x vectors and Matrix A on device.
    // EXERCISE: Use MemSpace and Layout.
    // Kokkos::View是Kokkos的通用的数组类型,类似std::shared_ptr
    // 1D可以视为向量类型,2D视为矩阵类型
    using ViewVectorType = Kokkos::View<double *, Layout, MemSpace>;
    using ViewMatrixType = Kokkos::View<double **, Layout, MemSpace>;

    // 定义需要的矩阵A和向量y,x
    // 传入的字符串作为lable唯一的确定对应的实例,在debug中非常有效
    ViewVectorType y("y", N);
    ViewVectorType x("x", M);
    ViewMatrixType A("A", N, M);

    // CUDA中的数据不能被CPU访问和读取,如果需要进行改动,需要先创建
    // CPU可访问的数据,也就是ViewVectorType::HostMirror
    ViewVectorType::HostMirror h_y = Kokkos::create_mirror_view(y);
    ViewVectorType::HostMirror h_x = Kokkos::create_mirror_view(x);
    ViewMatrixType::HostMirror h_A = Kokkos::create_mirror_view(A);
    
    // 初始化y 
    for (int i = 0; i < N; ++i)
    {
        h_y(i) = 1;
    }

    // 初始化x 
    for (int i = 0; i < M; ++i)
    {
        h_x(i) = 1;
    }

    // 初始化A 
    for (int j = 0; j < N; ++j)
    {
        for (int i = 0; i < M; ++i)
        {
        h_A(j, i) = 1;
        }
    }

    // 由CPU可访问空间转移到对应的并行空间
    Kokkos::deep_copy(y, h_y);
    Kokkos::deep_copy(x, h_x);
    Kokkos::deep_copy(A, h_A);

    // 计时器
    Kokkos::Timer timer;

    for (int repeat = 0; repeat < nrepeat; repeat++)
    {

        double result = 0;

        // 将yAx分解为y(j) * \sum_{i=0,M} A(j,i) * x(i)
        // range_policy(start,end,...)指定起始和结束index
        // KOKKOS_LAMBDA自动根据并行空间指定LAMBDA函数的初始化类型
        Kokkos::parallel_reduce("yAx", range_policy(0, N), 
        KOKKOS_LAMBDA (int j, double &update) {
        double temp2 = 0;

        for ( int i = 0; i < M; ++i ) {
        temp2 += A( j, i ) * x( i );
        }

        update += y( j ) * temp2; }, result);

        if (repeat == (nrepeat - 1))
        {
        std::cout << "  Computed result for " << N 
                    << " x " << M << " is " << result << std::endl;
        }
    }
    // 计算时间
    double time = timer.seconds();
    // 计算带宽
    double Gbytes = 1.0e-9 * double(sizeof(double) * (M + M * N + N));

    std::cout << "  N( " << N << " ) "
              << "  M( " << M << " ) "
              << "  nrepeat ( " << nrepeat << " ) "
              << "  problem( " << Gbytes * 1000 << " MB ) "
              << "  time( " << time << " s ) "
              << "  bandwidth( " << Gbytes * nrepeat / time << " GB/s )" << std::endl;
    }
    // 结束Kokkos
    Kokkos::finalize();

    return 0;
}

void checkSizes(int &N, int &M, int &S, int &nrepeat)
{

    if (S == -1 && (N == -1 || M == -1))
    {
    S = pow(2, 22);
    if (S < N)
        S = N;
    if (S < M)
        S = M;
    }

    if (S == -1)
    S = N * M;

    if (N == -1 && M == -1)
    {
    if (S > 1024)
    {
        M = 1024;
    }
    else
    {
        M = S;
    }
    }

    if (M == -1)
    M = S / N;

    if (N == -1)
    N = S / M;
}
\end{lstlisting}