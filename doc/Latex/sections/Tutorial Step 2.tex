\subsubsection{Tutorial Step 2}
这个教程主要介绍不同的储存顺序对于计算性能的影响,尤其是对于不同的架构
而言,对于CPU来说,由于其有缓存机制,每次读取都不会单独读取一个,而是
一次性读取相邻内存上的多个值放入缓存,那么在计算矩阵向量乘法的时候,
在遍历矩阵的列时,每次读取多个列值进行计算,整行计算完成后,步进到下一
行。对于GPU来说,由于存在合并访问机制,理想情况下合并效为100,
否则会浪费大部分的带宽,所以为列优先(GPU部分比较胡编乱造,只能说看个乐)。
\begin{lstlisting}
#include <Kokkos_Core.hpp>
#include <Kokkos_Timer.hpp>
#include <iostream>
#include <cstdio>
// 并行空间
using MemSpace = Kokkos::CudaSpace;

// 二维的数组类型,LayoutLeft -> 列优先,LayoutRight -> 行优先
// 对于CPU来说,矩阵Axb的过程中会大量读取和b中的值,为了提高
// 计算效率,如果Axb中读取一次就可以得到A(i,j)->A(i,j+n)无疑
// 会大大提高计算效率(Cache),对应的是行优先。进一步,如果一次将这读取的
// n个数一起计算,会有极为可观的收益(SIMD) 
// 对GPU来说,GPU核心非常多,由于存在合并访问,行优先会浪费很多的带宽,
// 一般是列优先
using left_type = Kokkos::View<double **, Kokkos::LayoutLeft, MemSpace>;
using right_type = Kokkos::View<double **, Kokkos::LayoutRight, MemSpace>;

using view_type = Kokkos::View<double *, MemSpace>;

// a.extent(0),a.extent(1)指的是View的维数大小,NxM的N和M
template <class ViewType>
struct init_view
{
    ViewType a;
    init_view(ViewType a_) : a(a_) {}

    using size_type = typename ViewType::size_type;

    KOKKOS_INLINE_FUNCTION
    void operator()(const typename ViewType::size_type i) const
    {
    for (size_type j = 0; j < static_cast<size_type>(a.extent(1)); ++j)
    {
        a(i, j) = 1.0 * a.extent(0) * i + 1.0 * j;
    }
    }
};

template <class ViewType1, class ViewType2>
struct contraction
{
    view_type a;
    typename ViewType1::const_type v1;
    typename ViewType2::const_type v2;
    contraction(view_type a_, ViewType1 v1_, ViewType2 v2_)
        : a(a_), v1(v1_), v2(v2_) {}

    using size_type = typename view_type::size_type;

    KOKKOS_INLINE_FUNCTION
    void operator()(const view_type::size_type i) const
    {
    for (size_type j = 0; j < static_cast<size_type>(a.extent(1)); ++j)
    {
        a(i) = v1(i, j) * v2(j, i);
    }
    }
};

// 向量内积
struct dot
{
    view_type a;
    dot(view_type a_) : a(a_) {}
    using value_type = double;
    KOKKOS_INLINE_FUNCTION
    void operator()(const view_type::size_type i, double &lsum) const
    {
    lsum += a(i) * a(i);
    }
};

int main(int argc, char *argv[])
{
    Kokkos::initialize(argc, argv);
    {
    int size = 10000;
    view_type a("A", size);

    // Define two views with LayoutLeft and LayoutRight.
    left_type l("L", size, 10000);
    right_type r("R", size, 10000);

    // Initialize the data in the views.
    Kokkos::parallel_for(size, init_view<left_type>(l));
    Kokkos::parallel_for(size, init_view<right_type>(r));
    // MPI_Barrier
    Kokkos::fence();

    Kokkos::Timer time1;
    Kokkos::parallel_for(size, contraction<left_type, right_type>(a, l, r));
    Kokkos::fence();
    double sec1 = time1.seconds();

    double sum1 = 0;
    Kokkos::parallel_reduce(size, dot(a), sum1);
    Kokkos::fence();

    Kokkos::Timer time2;
    Kokkos::parallel_for(size, contraction<right_type, left_type>(a, r, l));
    Kokkos::fence();
    double sec2 = time2.seconds();

    double sum2 = 0;
    Kokkos::parallel_reduce(size, dot(a), sum2);

    printf("Result Left/Right %f Right/Left %f (equal result: %i)\n", sec1,
            sec2, sum2 == sum1);    
    }
    Kokkos::finalize();
}
\end{lstlisting}