\subsubsection{Kokkos - Linear advection equation}
将从对流方程开始坐牢,这个方程的形式不算复杂,主要是使用连续FEM和离散FEM进行计算验证,
众所周知,对于这个双曲方程来说,连续的FEM计算结果会出现非常的的震荡,并且这种震荡不会
随着网格的细化而改善,主要原因是对流方程只约束了速度场的某一个方向导数上的值,但对于
对应的正交方向没有约束,导致如果在边界或这某处出现震荡,这个震荡会沿着特征线扩散到
起点处。

首先来看一下SUPG修正后的连续FEM计算方法,所有内容都只是图个乐,看个乐子。首先是方程形式:
\[
    \beta(x,y,z) \nabla u = f \qquad u = g \ , \quad u\in \partial \Omega_{-}
\]
在上述的方程中,$\beta(x,y,z)$是一个向量场,会随着位置的变化而变化,边界$\partial\Omega_{-}$
是入口边界,满足条件是$\{p\in\partial\Omega : \beta(p) \cdot \mathbf{n} < 0\}$,其实就是入口
边界,然后开始离散化,SUPG格式是迎风流线稳定,具体做法就是试探函数取为$v+\delta\beta\nabla v$,
,对于边界上的测试函数是完全看不明白,基本上就是推导半天得到的结论,取$\beta \cdot\mathbf{n} v$
作为边界测试函数,书上这么说,我也这么写,最不想看懂的一集。

所以上述说完基本上整个离散就结束了,最不想看懂的一集,别管怎么来的,先拿来用用看好不好,两边同时
乘以测试函数并进行积分,如下:
\[
\begin{aligned}
    \int_{\Omega} (v+\delta \beta \nabla v)\cdot \beta \nabla u - 
    \int_{\partial\Omega_{-}} \beta\cdot \mathbf{n} v u =
    \int_{\Omega} (v+\delta \beta \nabla v) f -
    \int_{\partial\Omega_{-}} \beta\cdot \mathbf{n} v g
\end{aligned}
\]
离散完成后就基本上按照上述进行计算即可,其实如果去掉边界的相,单纯的看原始方程的SUPG离散格式,
可以倒退到原始的强形式为$\beta \nabla v + \delta \nabla \cdot [\beta \cdot (\beta\cdot \nabla u)]=0$
,基本上就是多加了个扩散项,但又不是太大的扩散项,具体大小和网格的尺寸相关,也就是$\delta(h)$

对于DG方法来说,还是需要需要稳定项,但只是在内部边界上,
离散起来更为简洁,但是DG对于第一类边界条件要么加到弱形式里面,要么
加入惩罚,处理起来比较抽象。离散格式如下:
\[
    \begin{aligned}
        \int_{\Omega} v \beta \nabla u &= \int_{\partial\Omega} v \beta \cdot \mathbf{n} u -
        \int_{\Omega} u \beta \nabla v \\
        &= \int_{\partial\Omega_{-}} v \beta \cdot \mathbf{n} g + 
           \int_{\partial\Omega_{+}} v \beta \cdot \mathbf{n} u +
           \int_{\partial\Omega_{inner}} [v] \beta \cdot \mathbf{n} u^{upwind} -
           \int_{\Omega} u \beta \nabla v
    \end{aligned}
\]