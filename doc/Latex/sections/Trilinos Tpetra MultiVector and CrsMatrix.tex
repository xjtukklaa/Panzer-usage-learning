\subsubsection{Trilinos Tpetra MultiVector and CrsMatrix}
这段时间一直没更新主要是在看SUPG和DG,以及重新编译安装Trilinos,之前安装的版本缺少
一些内容,没有网格输入输出(STK)和自由度管理(Panzer)。给我折磨坏了,最新release版本有问题,方程
离散工具和自由度管理工具之间出现重复定义导致编译不通过,还得克隆下来master然后重新
安装,目前应该是完全的安装好了。

可能会比较迷惑的点在与Kokkos和Trilinos为啥会混合在一起,实际上Kokkos是Trilinos的
一个子项目,也是最核心的底层架构,主要负责矩阵向量表示,稀疏矩阵计算。Tpetra就是
以Kokkos为核心的线性代数核心包,因为网上的Trilinos介绍非常少,在此处简单的列举
安装上的核心库的主要作用。

\begin{itemize}
    \item Kokkos/KokkosKernels 核心底层
    \item Tpetra/Xpetra        线性代数表示/抽象接口
    \item Teuchos              提供参数文件/命令行参数,智能指针,MPI接口
    \item Amesos2              提供外部的稀疏直接求解器,MUMPS,SuperLU,UMPACK
    \item Anasazi              特征值计算
    \item Belos                提供各种迭代方法,包括无矩阵接口
    \item Galeri               提供有限差分法离散
    \item ifpack2              提供一般的预处理,SOR,Jacobian之类
    \item Intrepid2            提供旋度场,散度场,梯度场离散,基函数,正交规则
    \item ...  
\end{itemize}

这里面有必要着重提及Sacado和Muelu,后者是提供平滑聚类多重网格个几何多重网格的包,
前者是用于实现自动微分功能(尤其对于材料非线性,
多相流,敏度分析这些东西),你开始计算一个非线性材料的传热性能,辛辛苦苦推导出了对应
的Jacobian方程,输入到对应的牛顿方法中,进行计算。你觉得这套方案一定能够适用其他问题,
直到你遇到了多相流,参数敏度,你开始熬夜手搓,突然发现可以直接放弃大脑,拥抱现代c++。

Sacado提供正向直接微分和反向伴随微分,微分精度为机器精度,同时和Tpetra矩阵向量通过
模板结合,可以直接在cuda上跑。什么你说不是手动推导的没有灵魂,确实没有灵魂,但你手动
推导直接切线方程和伴随方程还没结束,这边可能就编写求解代码,边重载计算敏度了,折磨自己
不如折磨机器()

如果你安装了Sacado,以下是一个小测试案例

\begin{lstlisting}
#include <Sacado.hpp>
#include <iostream>
using fad_double = Sacado::Fad::DFad<double>;
int main() {
  fad_double a,b,c;
  a = 1; b = 2;
  a.diff(0,2);  // Set a to be dof 0, in a 2-dof system.
  b.diff(1,2);  // Set b to be dof 1, in a 2-dof system.
  c = 2*a+cos(a*b);
  double *derivs = &c.fastAccessDx(0); // Access derivatives
  double a_value = 1, b_value = 2;
  double dc_da = 2 - sin(a_value * b_value) * b_value;
  double dc_db = - sin(a_value * b_value) * a_value;
  std::cout << std::setprecision(16) << "Sacado : dc/da = " << derivs[0] << ", dc/db=" << derivs[1] << std::endl;
  std::cout << std::setprecision(16)<< "Hands  : dc/da = " << dc_da << ", dc/db=" << dc_db << std::endl;
}
\end{lstlisting}

其对应的输出结果如下:
\[
\begin{aligned}
Sacado &: dc/da = 0.1814051463486366, dc/db=-0.9092974268256817\\
Hands  &: dc/da = 0.1814051463486366, dc/db=-0.9092974268256817    
\end{aligned}
\]

对于大型的离散问题来说,使用自动微分对整个矩阵进行计算似乎是不是非常的现实,
但实际情况是,不会直接对整个稀疏矩阵进行计算,而是对每个单元刚度矩阵进行微分
再进行组合,使用c++模板重载实现在计算总体矩阵的时候同时计算Jacobian矩阵。对于
参数敏度更是如此,放弃智力,拥抱现代c++

以下代码是使用Galeri进行Laplace方程的有限差分离散,主要是需要搞明白矩阵向量的初始化。

\begin{lstlisting}
#include "Galeri_XpetraMaps.hpp"
#include "Galeri_MatrixTraits.hpp"
#include "Galeri_XpetraMatrixTypes.hpp"
#include "Galeri_XpetraProblemFactory.hpp"
#include "Teuchos_DefaultComm.hpp"
#include "Teuchos_ParameterList.hpp"

#define GO long long
#define Scalar int
#define LO int
#define Node Tpetra::KokkosCompat::KokkosOpenMPWrapperNode

using namespace Galeri;
int main(int argc, char *argv[])
{
    using Teuchos::RCP;
    using Teuchos::rcp;
    typedef Tpetra::Map<LO, GO, Node> Tpetra_Map;
    typedef Tpetra::CrsMatrix<Scalar, LO, GO, Node> Tpetra_CrsMatrix;
    typedef Tpetra::MultiVector<Scalar, LO, GO, Node> Tpetra_MultiVector;
    typedef Teuchos::ScalarTraits<Scalar> ScalarTraits;
#ifdef HAVE_MPI
    MPI_Init(&argc, &argv);
#endif
    // Create comm
    RCP<const Teuchos::Comm<int>> comm = Teuchos::DefaultComm<int>::getComm();
    // Here we create the linear problem
    //
    //   Matrix * LHS = RHS
    //
    // with Matrix arising from a 5-point formula discretization.
    std::string mapType = "Cartesian2D";
    auto mapParameters = Teuchos::ParameterList("Tpetra::Map");
    // dimension of the problem is nx x ny
    mapParameters.set("nx", 10 * comm->getSize());
    mapParameters.set("ny", 10);
    // total number of processors is mx x my
    mapParameters.set("mx", comm->getSize());
    mapParameters.set("my", 1);
    mapParameters.print();
    auto out = Teuchos::getFancyOStream(Teuchos::rcpFromRef(std::cout));
    try
    {
        // Creation of the map
        auto map = RCP{Galeri::Xpetra::CreateMap<Scalar, GO, Tpetra_Map>(mapType, comm, mapParameters)};
        // Creation of linear problem
        auto problem = Galeri::Xpetra::BuildProblem<Scalar, LO, GO, Tpetra_Map, Tpetra_CrsMatrix, Tpetra_MultiVector>("Laplace2D", map, mapParameters);
        // Build Matrix and MultiVectors
        auto matrix = problem->BuildMatrix();
        auto LHS = rcp(new Tpetra_MultiVector(matrix->getDomainMap(), 1));
        auto RHS = rcp(new Tpetra_MultiVector(matrix->getRangeMap(), 1));
        auto ExactSolution = rcp(new Tpetra_MultiVector(matrix->getDomainMap(), 1));
        ExactSolution->randomize(0, 100);
        LHS->putScalar(ScalarTraits::zero());
        matrix->apply(*ExactSolution, *RHS);
        matrix->describe(*out, Teuchos::EVerbosityLevel::VERB_EXTREME);
        LHS->describe(*out, Teuchos::EVerbosityLevel::VERB_EXTREME);
        RHS->describe(*out, Teuchos::EVerbosityLevel::VERB_EXTREME);
        ExactSolution->describe(*out, Teuchos::EVerbosityLevel::VERB_EXTREME);
        // at this point any LinearSolver can be used which understands the Tpetra objects. For example: Amesos2 or Ifpack2
    }
    catch (Galeri::Exception &rhs)
    {
        if (comm->getRank() == 0)
        {
            cerr << "Caught exception: ";
            rhs.Print();
#ifdef HAVE_MPI
            MPI_Finalize();
#endif
            return (EXIT_FAILURE);
        }
    }
#ifdef HAVE_MPI
    MPI_Finalize();
#endif
    return (EXIT_SUCCESS);
}
\end{lstlisting}